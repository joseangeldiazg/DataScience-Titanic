%?????????????????????????
% Nombre: capitulo5.tex  
% 
% Texto del capitulo 5
%---------------------------------------------------

\chapter{Conclusiones y v�as futuras}

En este cap�tulo final se estudian los resultados obtenidos a lo largo del trabajo y v�as futuras para aumentar a�n m�s el accuracy. Tambi�n se complementan las conclusiones que se han ido obteniendo a lo largo del trabajo. Por tanto, podr�amos resumir las conclusiones finales del trabajo en las siguientes:

\section{Conclusiones finales}

\begin{itemize}
	
	\item A lo largo del estudio hemos concluido que el dataset est� muy sujeto a la aleatoriedad del momento. Es un problema real con datos reales de personas que desafortunadamente perecieron en un desastre en el que estamos seguros el morir o no, no segu�a un patr�n concreto m�s all� del estudiado ya en este trabajo con las edades, los t�tulos y el sexo de la persona. 
	
	\item Es interesante mencionar como el aplicar un sencillo proceso de estudio de los datos en origen y ver si podemos obtener nuevas caracter�sticas ocultas de esto, realizando un minado sobre los datos originales, puede conseguir que nuestro sistema se comporte mucho mejor que usando simplemente el algoritmo de clasificaci�n m�s potente que conozcamos.  
	
	\item Seg�n hemos visto en los gr�ficos que nos ofrecen informaci�n acerca de como el \textbf{RandomForest} se comporta seg�n va realizando el aprendizaje, la variable que hemos creado en la secci�n de ingenier�a de caracter�sticas, \textbf{Title}, es la que m�s influye en la hora de hacer que nuestro resultado sea mejor y discernir entre si se sobrevive o no al desastre. La conclusi�n que llegamos con esto es que en esta �poca, el t�tulo estaba por encima de la clase o el dinero pagado, adem�s esta variable es una variable muy importante que aglutina en un solo factor informaci�n acerca del Sexo y la clase. 
	
	\item Aunque a priori no hay outliers o ruido en el problema, hemos comprobado como realizando una uni�n de los valores que salen de lo normal por su media, obtenemos mejores resultados. Respecto al ruido, como hemos comentado en la secci�n de XGBoost, hay cierto concepto de ruido entre caracter�sticas e instancias que deber�a ser mejorado para que este algoritmo funcionara mejor. 
	
\end{itemize}

\section{V�as futuras}

El estudio llevado a cabo es uno de los mas exhaustivos vistos sobre el problema. A�na las partes importantes de los dos mejores tutor�ales para este problema y sinceramente es complicado obtener mejores resultados. Como v�as futuras cabria la posibilidad de usar algoritmos evolutivos que nos ayuden en el proceso de selecci�n de caracter�sticas evitando as� posiblemente los ruidos que hacen que el algoritmo XGBoost no se comporte como debe. Por otro lado, cabr�a la posibilidad tambi�n de ajustar este algoritmo usando t�cnicas de oversampling para equilibrar las clases, ya que est� constatada en la literatura la potencia de union de ambos algoritmos. 

\pagebreak
\clearpage
%---------------------------------------------------